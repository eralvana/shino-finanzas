\documentclass[12pt]{article}

\usepackage[T1]{fontenc}
\usepackage{lmodern}

\usepackage{amssymb}
\usepackage{amsmath}
\usepackage{amscd}
\usepackage{amsthm}
\usepackage[utf8]{inputenc}
\usepackage[spanish,mexico]{babel}
\usepackage{enumerate}
\usepackage{pgf,tikz}
\usepackage{makeidx}
\usetikzlibrary{arrows}
\usepackage{multicol}

\usepackage[colorlinks=true, linkcolor=blue, urlcolor=blue, citecolor=red]{hyperref}

\usepackage{graphicx}
\usepackage{wrapfig}

\usepackage{setspace}

%\voffset=-2 cm
%\hoffset=-2 cm
%\textwidth=19 cm
%\textheight=22 cm


\theoremstyle{definition}
\newtheorem{teo}{Teorema}[section]
\newtheorem{df}[teo]{Definición}
\newtheorem{ej}[teo]{Ejercicio}
\newtheorem{prop}[teo]{Proposición}
\newtheorem{lema}[teo]{Lema}
\newtheorem{cor}[teo]{Corolario}
\newtheorem{obs}[teo]{Observación}
\newtheorem{ejp}[teo]{Ejemplo}

\newcommand{\N}{\mathbb N}
\newcommand{\Pol}{{\cal P}}
\newcommand{\F}{{\cal F}}

\newenvironment{pba}{\noindent\textbf{\textcolor{blue}{Prueba:}}}{\begin{flushright}
$\square$ \end{flushright}}
\newenvironment{dem}{\noindent\textbf{\textcolor{blue}{Demostración}:}}{\begin{flushright}
\rule{1ex}{1ex} \end{flushright}}



\title{Proyecto \textbf{Finanzas}}
\author{Braulio Aguilar Simón \and Jorge Leopoldo Caballero Arredondo \and Ernesto Alejandro Vázquez Navarro}

\begin{document}

\maketitle

\abstract{El presente proyecto intenta representar los requerimientos para el proyecto \textbf{Finanzas}.}

\section{Tablas}

Se requerirá de la siguiente estructura de tablas:
\begin{itemize}
\item Una tabla de \textit{Ingresos general} con las siguientes columnas:
\begin{itemize}
\item Nombre (Texto)
\item Tipo de ingreso (Dos opciones: eventual-periódico)
\item Fecha (Formato de fecha)
\item Monto (Formato de cantidad a dos dígitos)
\item Concepto (Texto)
\item Nota (Texto para añadir información adicional)
\end{itemize}

\item Distintas tablas de \textit{Ingresos particulares} (ejemplos: Tarjeta de crédito, tarjeta de débito, dinero en el colchón, dinero en mi bolsillo, etcétera) con las siguientes columnas:
\begin{itemize}
\item Nombre (Texto)
\item Fuente de Ingreso (Texto: Nombre de la tabla \textit{Ingresos general})
\item Fecha (Formato de fecha)
\item Monto (Formato de cantidad a dos dígitos)
\item Concepto (Texto)
\item Nota (Texto para añadir información adicional)
\end{itemize}

\item Distintas tablas de \textit{Egresos} (ejemplo: Tarjeta de crédito, gastos, compras, etcétera) con las siguientes columnas:
\begin{itemize}
\item Nombre (Texto)
\item Tipo de ingreso (se alimentará de la Tabla de \textit{Ingresos particulares})
\item Fecha (Formato de fecha)
\item Monto (Formato de cantidad a dos dígitos)
\item Concepto (Texto)
\item Pagos (Número entero que por defecto es uno)
\item Nota (Texto para añadir información adicional)
\end{itemize}

\end{itemize}

\end{document}


